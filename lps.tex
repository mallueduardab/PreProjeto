\chapter{LINHAS DE PRODUTOS DE SOFTWARE - ESTADO DA ARTE}
\label{cap:lps}

\section{Considerações iniciais}

Uma Linha de Produtos de Software (LPS) é formada por um conjunto de sistemas de softwares que foram desenvolvidos com base em um código comum e buscam atingir um segmento de mercado específico \cite{apel2016feature}. A gerência desse recurso é feita por meio de um modelo de árvore, operadores lógicos e restrições que podem ser implementados por tecnologias baseadas em composição \cite{Vale2015}.


Tendo em vista tais considerações, esse capítulo está estruturado da seguinte maneira. Na Seção 2.1 são descritas as considerações iniciais sobre linha de produtos de software e uma visão geral sobre o assunto. Na Seção 2.2 são abordadas características, variabilidades e comunalidades em LPS. A Seção 2.3 aborda a evolução de linha de produtos. A seção 2.4 introduz o ambiente de clonagem de linha de produtos. A seção 2.5 Abordam as considerações finais levantadas no desenvolvimento desse capítulo.
	
\section{Características e Produtos}


O termo Linha de Produtos de Software (LPS) surgiu da necessidade da indústria em promover adaptações a um produto, de acordo com as necessidades particulares de cada cliente \cite{apel2016feature}. Isso possibilitou a variabilidade de produtos compondo uma família, gerada a partir de um conjunto de características em comum \cite{Laguna2013}. 

O reúso de artefatos de software 


{Para formar a base de uma linha de produção, são necessários artefatos e recursos, denominados ativos base (\textit{core assets}), que incluem por exemplo componentes, modelo de domínio, requisitos, especificações, arquitetura e outros}
 
Quanto à estrutura, uma LPS é formada por engenharia de domínio, engenharia de aplicação e evolução da própria LPS. A engenharia de domínio é responsável pelo domínio de funcionalidades bem como a criação e manutenção do núcleo da LPS. Se tratando de engenharia de aplicação consiste basicamente no desenvolvimento do produto. 



\subsection{Variabilidades}

O que diferem os produtos gerados em uma LPS uns dos outros, são as variabilidades acrescentadas a cada um deles. Tais variabilidades são responsáveis por acrescentar características específicas a cada produto.

A implementação de uma linha de produtos segue uma estrutura de árvore que representa o modelo de características (\textit{feature model}). Os nós obrigatórios (obrigatoriedades) são acrescidos de características (\textit{features}) variáveis (variabilidades) e as várias combinações possíveis dessas, possibilita a geração de vários produtos.

Variabilidades podem ser obrigatórias, opcionais ou alternativas. Nesse contexto, variabilidades obrigatórias estão presentes em todos os produtos da LPS. As opcionais pertencem a alguns produtos e as alternativas, dado um conjunto de características, possibilita a escolha de somente uma \cite{colanzi2014abordagem}.

*Seção ainda sem resultados/em pesquisa como parte do cronograma com o orientador.*


\section{Processo de Desenvolvimento}



\section{Evolução}

*Seção ainda sem resultados/em pesquisa como parte do cronograma com o orientador.*


\section{LPS Orientada a Características}


\section{Clonagem de código}

Linhas de Produtos de Software foram propostas como uma abordagem melhor estruturada para a  reutilização de artefatos de código-fonte entre um conjunto de sistemas com características semelhantes, conhecidos como família .	

*Seção ainda sem resultados/em pesquisa como parte do cronograma com o orientador.*

\section{Considerações finais}

Este capítulo descreve o cenário e configurações de Linhas de Produtos de Software. As informações obtidas até aqui serão completadas e continuado o capítulo como próximo passo para o texto de pré projeto.
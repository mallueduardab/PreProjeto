\chapter{Metodologia de Pesquisa}
\label{cap:metodologiapesquisa}

Metodologia determina formas a serem utilizadas para reunir dados necessário para o êxito do trabalho, através de técnicas de coleta e análise de dados \cite{moresi2003metodologia}. 

O restante desse capítulo está organizado da seguinte forma. A Seção 2.1 classifica a pesquisa de acordo com \citeauthor{moresi2003metodologia}, \citeyear{moresi2003metodologia}. Na Seção 2.2, são apresentadas as etapas de desenvolvimento do projeto bem como suas especificações no processo. 

\section{Classificação da Pesquisa}

Segundo \citeauthor{moresi2003metodologia} \citeyear{moresi2003metodologia}, a pesquisa pode ser classificada quanto a:

\begin{itemize}
	\item \textbf{Natureza}: do ponto de vista de sua natureza, a pesquisa classifica-se em (a) básica, que tem por objetivo gerar conhecimentos úteis para obter avanços na ciência, sem previsão de aplicação prática e envolve verdades e interesses universais, ou (b) aplicada, que tem por objetivo gerar conhecimento, aplicáveis na prática, buscando solução para problemas específicos e envolve verdades e interesses locais. Este trabalho pode ser classificado como \textbf{pesquisa aplicada}, pois visa o desenvolvimento de uma abordagem de detecção de clones de código em Linhas de Produtos de Software.
	
	\item  \textbf{Abordagem:} do ponto de vista da abordagem do problema, a pesquisa classifica-se em (a) quantitativa, que tem por objetivo traduzir informações em números para classifica-las e analisa-las através do uso de técnicas de estatística, ou (b) qualitativa, que utiliza de interpretação de fenômenos e atribuição de significado, sendo considerada descritiva. Este trabalho pode ser classificado como \textbf{pesquisa quantitativa} pois utiliza de recursos e estatística para análise dos resultados.
	
	\item \textbf{Finalidade}: do ponto de vista da finalidade, a pesquisa pode ser (a) exploratória, que a investigação acontece onde existe pouco conhecimento acumulado e sistematizado,  e/ou (b) descritiva, onde a pesquisa é direcionada a mostrar características de determinados fenômenos ou populações, e/ou (c) explicativa, que tenta tornar algo compreensível, através da justificativa dos motivos, e/ou (d) metodológica, que é referente a elaboração de instrumentos de captação ou de manipulação da realidade, e/ou (e) intervencionista, que objetiva principalmente interferir na realidade estuda visando modifica-la. Este trabalho pode ser classificado como \textbf{pesquisa exploratória e metodológica} visto que não foram encontrados estudos sobre detecção de clones de código em linha de produtos de software orientada a características em um mapeamento sistemático da literatura realizado, e que o método de detecção proposto disporá de uma ferramenta que semi-automatiza o processo. 
	
	\item \textbf{Meios de investigação:} do ponto de vista das maneiras de busca pela informação, a pesquisa pode ser (a) de campo, que consiste na investigação empírica no local onde ocorre/ocorreu um fenômeno ou que disponibiliza elementos que o explica, ou (b) de laboratório, que a experiência é realizada em local restringido, ou (c) telematizada, que utiliza computador e telecomunicações, ou (d) documental, que faz investigação de documentos, ou (e) bibliográfica, que é o estudo sistematizado desenvolvido baseando-se em material disponível ao publico em geral, ou (f) experimental, que é investigação empírica utilizando variáveis independentes de forma a manipula-las e controla-las para observar variações que as mesmas surtem em variáveis dependentes, ou (g) \textit{ex post facto}, que refere-se a um fato já ocorrido, ou (h) participante, que introduz a fronteira pesquisador/pesquisado ao contexto do problema investigado, ou (i) pesquisa-ação, que supõe intervencionar participativamente na realidade social, (j) estudo de caso, que é delimitado a uma ou poucas unidades e tem caráter de detalhamento. Este trabalho pode ser classificado como \textbf{pesquisa bibliográfica} pois realiza o estudo sistemático de materiais contidos em bibliotecas digitais, livros e outros.
	
\end{itemize}

\section{Método de Pesquisa}

O presente trabalho iniciou-se em março de 2018 e tem como previsão o término para fevereiro de 2020. Neste trabalho, o objetivo é desenvolver uma abordagem híbrida de detecção de clones de código em Linhas de Produtos de Software Orientada a Características. Para realizar a análise dessa abordagem, serão abrangidos os contextos de análise de desempenho e [COMPLETAR - N TO LEMBRANDO A DEFINIÇÃO CERTA]. Logo, a metodologia a ser utilizada nesse trabalho segue as seguintes etapas:

\begin{itemize}
	\item[1] \textbf{Revisão da Literatura:} nessa etapa foi realizada a revisão da literatura afim de conhecer os campos de estudo e obter informações importantes ao escopo do trabalho. No decorrer desta, foi fundamentada a teoria acerca de (i) Linhas de Produtos de Software e (ii) Clonagem de Código. Para essa ultima foi realizado um estudo de mapeamento sistemático da literatura que reúne ferramentas, técnicas, processos e outros, sobre detecção de código clonado. 
	
	\item[1] \textbf{Revisão da Literatura:}
\end{itemize}






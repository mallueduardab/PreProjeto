\chapter{TRABALHOS RELACIONADOS	}
\label{cap:trabalhosrelacionados}

Nesta seção são é descritas contextos de aplicação de detecção de clones em softwares em conjunto ou não com linhas de produtos de software, de modo a descrever a relação deste trabalho com outros já propostos.


Em um trabalho (\cite{alexandremartinspaiva2016}), há a proposta de desenvolvimento e de avaliação de um novo método de detecção de clone de código baseado em sequências similares de chamada de métodos. A ferramenta desenvolvida para detecção (\texttt{McSheep} é comparada com a ferramenta PMD\footnote{PMD é uma ferramenta de análise que varre código Java e dispõe da implementação da análise CPD (Copy/Paste Detector), que detecta código duplicado.} para detectar clones não identificados. Participaram de uma pesquisa 25 desenvolvedores para avaliar o método e mais de 90\% deles concordam com sua validade para detecção de clones. O resultado da comparação das ferramentas mostra que ambas podem ser utilizadas de forma híbrida, pois existe um conjunto interseção de casos detectados e, para cada ferramenta, um conjunto exclusivo.

Em outro trabalho (\cite{Schulze2011a}), há descrição de razões para existência de clones relacionados à programação orientada a características e como lidar com eles. Para isso, aplicar refatorações em clones sintáticos para remoção de clones em LPS's é a abordagem proposta, visando à permanência dos benefícios de programação orientada a características como mais coesão e reutilização de códigos de características. Os resultados foram: i) cerca de 10\% a 15\% do código das dez LPS's analisadas são clones; ii) entre 9\% a 12\% desses podem ser refatorados com base na sintaxe; iii) mais ou menos 9\% dos clones são relacionados a POC e 2\% a 3\% relacionados a POO; e iv) em todas as etapas de análise, a quantidade de clones é maior em LPS's desenvolvidas do zero do que em LPS's decompostas de sistemas legados.

Assim como os trabalhos citados, essa investigação propõe a detecção de clones de código em código. Mas, há diferenças a serem consideradas, tais como, a abordagem de como é feita a detecção de clones em paradigmas diferentes (programação orientada a objetos e programação orientada a características), propor uma abordagem, (semi)automatizar a abordagem proposta (apoio computacional) para detectar clones diretamente na LPS e analisar os resultados da abordagem com relação à manutenibilidade.